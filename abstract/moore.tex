\documentclass[10pt,letterpaper,notitlepage,twocolumn]{article}

\usepackage[top=0.75in, bottom=0.75in, left=0.5in, right=0.5in]{geometry}
\usepackage{times}
\usepackage{graphicx}

\title{Identification of human control during perturbed walking}
\author{
  Jason K. Moore, Sandra K. Hnat, Antonie J. van den Bogert\\
  Human Motion and Control Laboratory, Cleveland State University, Cleveland, Ohio, USA\\
  Email: j.k.moore19@csuohio.edu Web: http://hmc.csuohio.edu
}
\date{}

\begin{document}
\pagenumbering{gobble} % removes page numbers

\maketitle

\section*{Introduction}
%
Gait related powered prosthetics have begun to emerge in research and more
recently commercial use. These devices include a variety of sensors and
actuators than can be coupled by a control system to provide more natural gait
and eventually balance to people with lower extremity disabilities. In
particular, there are a few light weight lower extremity exoskeletons on the
market that assist in walking but lack gait that resembles an able bodied
human. Our intent is to identify a simple linear controller from a large set of
data collected from able bodied walkers under prescribed longitudinal and/or
lateral perturbations with the power prosthetics application in mind.
%
\section*{Methods}
%
Herein were present results from data collected from a single subject (age: 25,
mass: 101 kg, height: 187 cm) walking at 1.2 m/s on a treadmill (ForceLink
R-Mill) while being longitudinally perturbed, i.e. a random white noise with
5\% std around a nominal belt speed. We collect data for four minutes at 100 Hz
which includes about ~200 steps. We compute the ankle plantarflexion, knee
flexion, and hip flexion angles, rates, and moments using basic 2D inverse
dynamics \cite{Winter}. % TODO : Add references.

We section the inverse dynamics time series into steps based on the right
foot's heel strike and interpolate 20 evenly spaced data points for each series
along the gait cycle. We assume a simple proportional derivative controller
that generates the joint torques given the joint angles and rates that fits
this form:
%
\begin{equation}
  m_m(\phi) = m_0(\phi) + \mathbf{K}[s_0(\phi) - s_m(\phi)]
\end{equation}
%
where $\phi$ is the time in the right leg gait cycle, $m_m(\phi)$ is a vector
of measured joint torques, $m_0(\phi)$ are the joint torques needed if the
feedback error is zero, $\mathbf{K}$ is the gain matrix which multiplies the
error in the joint angles and rates with respect to the set point, $s_0(\phi)$.
We then rearrange the equation to make it linear in the measured sensors and a
reference torques, $m^*(\phi)$.
%
\begin{equation}
  m_m(\phi) = m^*(\phi) - \mathbf{K}s_m(\phi)
\end{equation}
%
Given enough measured data, $m_m(\phi)$ and $s_m(\phi)$, the reference torques
$m^*(\phi)$ and the gains $\mathbf{K}$ can be solved for using linear least
squares. The externally applied longitudinal perturbations are important for
reducing the uncertainty in the identified parameters.
%
\section*{Results}
%
Here we show an example results from a controller structure which is limited
such that the joint torques are only generated from the error in the sensors
from that same joint. Figure \ref{fig:gains} shows the estimates of the
scheduled gains with respect to the percent gait cycle (right heel strike at
0\%). Figure \ref{fig:fit} shows an example prediction of the measured ankle
plantarflexion torque in the right leg by the identified control model with a
variance accounted for [VAF] of 76.8\%.
%
\begin{figure}
  \begin{center}
    \includegraphics[width=\columnwidth]{fig/gains.pdf}
    \label{fig:gains}
    %\caption{Needs one.}
  \end{center}
\end{figure}
%
\begin{figure}
  \begin{center}
    \includegraphics[width=\columnwidth]{fig/fit.pdf}
    \label{fig:fit}
    %\caption{Needs one.}
  \end{center}
\end{figure}
%
\section*{Discussion}
%
We are able to identify a simple linear controller that exhibits larger gains
in the stance phase than in the swing phase, although there is little
indication of similar gain patterns in the right and left legs. The controller
is also able to predict the measured joint torques with high accuracy in all
joints, $>$ 60\% VAF. Future work will include non-linear control structures
with reflexive time delays and data collected from a sample population.
%
\section*{Acknowledgments}

This research was funded by the Ohio's Wright Center for Sensor Systems
Engineering and the Parker Hannifin Corporation.

\end{document}
