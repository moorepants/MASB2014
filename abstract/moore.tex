\documentclass[10pt,letterpaper,notitlepage,twocolumn]{article}

\usepackage[top=0.75in, bottom=0.75in, left=0.5in, right=0.5in]{geometry}
\usepackage{times}
\usepackage{graphicx}

\title{\bf Identification of human control during perturbed walking}
\author{
  Jason K. Moore, Sandra K. Hnat, Antonie J. van den Bogert\\
  Human Motion and Control Laboratory, Cleveland State University, Cleveland, Ohio, USA\\
  Email: j.k.moore19@csuohio.edu Web: http://hmc.csuohio.edu
}
\date{}

\begin{document}
\pagenumbering{gobble} % removes page numbers

\maketitle

\section*{Introduction}
%
Recent research and commercial activity have shown that gait related powered
prosthetics will play an important role in assisting humans with disabilities.
These devices include a variety of sensors and actuators than can be coupled by
a control system to provide gait assistance. In particular, there are light
weight lower extremity exoskeletons on the market that assist in walking but
lack gait that resembles that of an able-bodied human. Our intent is to
identify a simple linear controller from a large set of data collected from
able-bodied walkers under prescribed longitudinal and/or lateral perturbations
with powered prosthetic applications in mind.
%
\section*{Methods}
Herein we present results from data collected from a single subject (age: 25,
mass: 101 kg, height: 187 cm) walking around 1.2 m/s on a treadmill (ForceLink
R-Mill) while being longitudinally perturbed, i.e. a random white noise with
5\% std around the nominal belt speed. We collect data for four minutes at 100
Hz which includes about 200 steps. We compute the ankle plantarflexion, knee
flexion, and hip flexion angles, rates, and moments using 2D inverse dynamics.
%
\begin{figure}[hbt]
  \begin{center}
    \includegraphics[width=\columnwidth]{fig/gains.pdf}
    \caption{Scheduled gains for right (blue) and left (red) legs.}
    \label{fig:gains}
  \end{center}
\end{figure}

We section the joint angle, rate, and torque time series into steps based on
the right foot's heel strike and interpolate 20 evenly spaced data points from
each series along the gait cycle. We assume a simple proportional derivative
controller that generates the joint torques given the joint angles and rates
that fit this form:
%
\begin{equation}
  m_m(\varphi) = m^*(\varphi) - \mathbf{K}s_m(\varphi)
\end{equation}
%
where $\varphi$ is the time in the right leg gait cycle, $m_m(\varphi)$ is a vector
of measured joint torques, $m^*(\varphi)$ are the reference joint torques,
$\mathbf{K}$ is the gain matrix which multiplies the measured sensor vector.
This equation is linear in gains and the reference torques, so given enough
measured data the reference torques and the gains can be solved for using
linear least squares. The externally applied perturbations are important for
reducing the uncertainty in the identified parameters.
\section*{Results}
%
Here we show an example result from a controller structure which is limited
such that the joint torques are only generated from the error in the sensors
from that same joint. Figure \ref{fig:gains} shows the estimates of the
scheduled gains with respect to the percent gait cycle in each leg. Figure
\ref{fig:fit} shows an example prediction of the measured ankle plantarflexion
torque in the right leg by the identified control model with a variance
accounted for [VAF] of 77\%.
%
\begin{figure}[b]
  \begin{center}
    \includegraphics[width=\columnwidth]{fig/fit.pdf}
    \caption{Predicted torque compared to independent validation data.}
    \label{fig:fit}
  \end{center}
\end{figure}
%
\section*{Discussion}
%
We are able to identify a simple linear controller that exhibits larger gains
in the stance phase than in the swing phase and there is indication of similar
gain patterns in the right and left legs. The controller is also able to
predict the measured joint torques with greater than 60\% VAF in all joints.
%
%\section*{Acknowledgments}
%
%This research was funded by the Ohio's Wright Center for Sensor Systems
%Engineering and the Parker Hannifin Corporation.
\end{document}
