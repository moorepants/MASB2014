\documentclass[10pt,letterpaper,notitlepage,twocolumn]{article}

\usepackage[top=0.75in, bottom=0.75in, left=0.5in, right=0.5in]{geometry}
\usepackage{times}
\usepackage{graphicx}

\title{\bf Identification of human control during perturbed walking}
\author{
  Jason K. Moore, Sandra K. Hnat, Antonie J. van den Bogert\\
  Human Motion and Control Laboratory, Cleveland State University, Cleveland, Ohio, USA\\
  Email: j.k.moore19@csuohio.edu Web: http://hmc.csuohio.edu
}
\date{}

\begin{document}

\pagenumbering{gobble} % removes page numbers
\maketitle

\section*{Introduction}
%
Recent research and commercial activity have shown that gait-related, powered
prosthetics will play an important role in assisting humans with disabilities.
These devices include a variety of sensors and actuators than can be coupled by
a control system to provide gait assistance. However, the available
lightweight, lower extremity exoskeletons lack gait that resembles an
able-bodied human. To improve the gait of powered prosthetics, our intent is to
identify a simple, linear controller from a large set of data collected from
able-bodied subjects being perturbed by random longitudinal and/or lateral
forces.
%
\section*{Methods}
The example data herein was collected from a single subject (age: 25, mass: 101
kg, height: 187 cm) walking on instrumented treadmill (VG005-A, Motek Medical,
Amsterdam, Netherlands). The subject was longitudinally perturbed using random
white noise with 5\% std around a nominal 1.2 m/s belt speed. Data was recorded
for 4 minutes at 100 Hz, which included approximately 200 steps of walking.
Ankle plantarflexion, knee flexion, and hip flexion angles, rates, and moments
were computed using 2D inverse dynamics.
%
\begin{figure}[hbt]
  \begin{center}
    \includegraphics[width=\columnwidth]{fig/gains.pdf}
    \caption{Scheduled gains for right (blue) and left (red) legs.}
    \label{fig:gains}
  \end{center}
\end{figure}

Joint angle, rate, and torque time series were sectioned into steps based on
the right foot's heel strike. Subsequently, 20 evenly spaced data points from
each series were interpolated along the gait cycle. We assume a simple
scheduled proportional derivative controller that generates the joint torques
given the joint angles and rates that fits the following form.
%
\begin{equation}
  \mathbf{m}(t) = \mathbf{m}^*(\varphi(t)) -
  \mathbf{K}(\varphi(t))\mathbf{s}(t)
\end{equation}
%
where $t$ is an instance of time, $\varphi(t)$ is the phase in the right leg
gait cycle, $\mathbf{m}(t)$ is a vector of joint torques,
$\mathbf{m}^*(\varphi(t))$ is a vector of the reference joint torques,
$\mathbf{K}(\varphi(t))$ is the gait phase scheduled gain matrix which
multiplies the vector of joint angles and rates, $\mathbf{s}(t)$. This equation
is linear in the gains and the reference torques. Given sufficient joint angle,
rate, and torque measurements, the reference torques and the gains can be
solved for using linear least squares.
%
\section*{Results}
%
Here we present an example result from a controller structure which is limited
to joint torque generation only from error in the sensors from the same joint.
Figure \ref{fig:gains} shows the estimates of the scheduled gains with respect
to the percent gait cycle in each leg. Figure \ref{fig:fit} demonstrates an
example prediction of the measured ankle plantarflexion torque in the right leg
by the identified control model.
%
\begin{figure}[b]
  \begin{center}
    \includegraphics[width=\columnwidth]{fig/fit.pdf}
    \caption{Predicted torque compared to independent validation data.}
    \label{fig:fit}
  \end{center}
\end{figure}
%
\section*{Discussion}
%
We are able to identify a simple linear controller that exhibits larger gains
in the stance phase than in the swing phase. Additionally, similar gain
patterns in the right and left legs are observed. The controller is capable of
predicting the measured joint torques with greater than 60\% VAF in all joints.
%
%\section*{Acknowledgments}
%
%This research was funded by the Ohio's Wright Center for Sensor Systems
%Engineering and the Parker Hannifin Corporation.
\end{document}
